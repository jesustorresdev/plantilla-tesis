%%
%% capitulo01.tex - Memoria de la tesis
%%
%%   Copyright 2009-2010 Jesús Torres <jmtorres@ull.es>
%%
%% Esta obra está bajo licencia Creative Commons Reconocimiento 3.0 Unported
%%
\chapter{Capítulo 1}\label{ch:capitulo}

\section{Sección 1 del capítulo 1}
Lorem ipsum dolor sit amet, consectetur adipiscing elit. Vestibulum faucibus
sollicitudin facilisis. Donec sollicitudin augue eu lectus gravida mollis.
Nunc lacinia imperdiet posuere. Curabitur augue nulla, placerat sit amet
pellentesque a, venenatis id ante. Ut eget neque eros, sit amet ornare risus.
Quisque quis blandit tortor. Donec nunc justo, consectetur ac tempor eget,
dignissim ut arcu. Nam eu justo vel est imperdiet dignissim. Phasellus
vulputate augue nec est tincidunt fermentum. Sed hendrerit nisl sed turpis
blandit vestibulum. Vestibulum vitae lectus ac ligula varius iaculis. Maecenas
vitae lorem arcu, nec convallis leo. Nulla cursus, odio sit amet semper
hendrerit, odio magna dignissim massa, ut mattis justo felis at ante. Class
aptent taciti sociosqu ad litora torquent per conubia nostra, per inceptos
himenaeos.

Nunc tincidunt fringilla dictum. In ac dolor et nibh viverra sodales a vitae
purus. Curabitur ultricies bibendum dolor eget lacinia. Duis fermentum neque
sit amet purus tempus in porta mauris posuere. Sed sit amet nulla odio. Nam
sit amet fermentum lorem. Mauris accumsan suscipit lorem. Suspendisse sed
nulla mi, id viverra tellus. Ut tristique egestas eros, ut molestie risus
aliquam nec. In aliquet, mi sit amet fringilla cursus, nisl velit tincidunt
lacus, in euismod lorem erat sit amet urna. Quisque ultrices luctus odio
sollicitudin malesuada.

\begin{algorithm}
\begin{algorithmic}[1]
\REQUIRE Complejo simplicial $K=\{\sigma_1, \dots, \sigma_n \}$ no vacío. \label{lin:lineaRara}
\ENSURE \TRUE{} si $K$ es contráctil y \FALSE{} en caso contrario.
\WHILE {$K \neq \{ \langle v \rangle \}$}
\STATE Elegir un símplice $\sigma_i$ de $K$ que sea maximal y que contenga una cara libre $\delta\sigma_i$.
\IF{no hay ningún símplice de esas características}
\RETURN \FALSE
\ELSE
\STATE $K \leftarrow K \setminus \{\sigma_i, \delta\sigma_i\}$
\ENDIF
\ENDWHILE
\RETURN \TRUE
\end{algorithmic}
\caption{Contracción de caras libres}\label{alg:algoritmoRaro}
\end{algorithm}

Ut pretium, eros consequat gravida viverra, purus neque ornare odio, id
pretium erat mauris ut justo. Cras vel tellus ut arcu faucibus condimentum
mattis eget justo. Phasellus ornare tellus sed tellus ornare pretium.
Phasellus non nunc elit, id aliquam leo. Integer tempus ligula at lectus
laoreet sit amet sollicitudin odio convallis. Donec sodales nisi ac eros
faucibus eu gravida magna eleifend. Cras in nunc sed massa congue egestas a
eget massa. Lorem ipsum dolor sit amet, consectetur adipiscing elit. Maecenas
egestas, enim vel faucibus consequat, nulla sapien mattis dui, et lobortis est
metus nec augue. Quisque rhoncus pellentesque hendrerit. Donec tristique metus
quis ligula hendrerit hendrerit. Nulla pretium tincidunt sagittis. Nullam
luctus fermentum nibh venenatis sollicitudin. Nulla sem risus, fringilla id
sollicitudin sit amet, elementum laoreet nisi. Ut nec mauris ac elit pretium
congue eget vel ipsum.

In et elit velit. Donec consectetur auctor tortor, et lobortis felis
sollicitudin dictum. Nunc imperdiet massa et sem eleifend molestie. Nam sed
nunc vitae orci cursus iaculis. Aenean eu bibendum risus. Vestibulum et ipsum
arcu. Nam sit amet libero ut enim pellentesque interdum. Suspendisse semper
rutrum facilisis. Duis auctor, lectus ac dignissim bibendum, sem magna
adipiscing nunc, sit amet tincidunt ligula leo in lorem. Sed convallis quam et
felis sagittis ultrices. Nulla lorem dui, consequat ac pulvinar quis, mollis
et purus. Fusce ullamcorper, turpis quis ornare tincidunt, lorem arcu
pellentesque odio, et elementum magna velit quis nulla. In non dictum sapien.
Phasellus placerat, urna dignissim posuere bibendum, turpis nibh porta ante,
in ultricies leo mauris eu ipsum. Integer pulvinar mattis euismod. Ut sit amet
orci vulputate nisl malesuada auctor ac aliquet lacus. Curabitur porttitor
nulla ut augue rhoncus sollicitudin. In ligula nisl, semper vel imperdiet at,
elementum vel dui.
