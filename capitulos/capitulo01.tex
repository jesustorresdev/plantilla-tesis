%%
%% capitulo01.tex - Memoria de la tesis
%%
%%   Copyright 2009-2018 Jesús Torres <jmtorres@ull.es>
%%
%% Esta obra está bajo licencia Creative Commons Reconocimiento 4.0 Internacional
%%
\chapter{Capítulo 1}\label{ch:capitulo}

\section{Sección 1 del capítulo 1}\label{sec:seccion}

Dentro del documento se pueden describir algoritmos, que como son \emph{floats}
pueden aparecer en cualquier parte. Por eso se deben referenciar, como se puede ver
en \vautoref{alg:algoritmoRaro}, incluso a \autoref{alg:lin:output} y otras
líneas concretas. También podemos hablar por primera vez de \ac{LM} para que al volver
a usar \ac{LM} como acrónimo ocurra la magia. Y podemos incluir figuras ~---véase \vautoref{fig:unaFigura}~---.

\begin{algorithm}
\caption{Búsqueda de números pares.}\label{alg:algoritmoRaro}
\begin{algorithmic}[1]
\REQUIRE Un conjunto $v$ no vacío de enteros.
\ENSURE El primer elemento par de $v$.
\FOR{$i$ in $v$}\label{alg:lin:output}
\IF{$i$ es par}
\RETURN $i$
\ENDIF
\ENDFOR
\end{algorithmic}
\end{algorithm}

\begin{figure}
  \centering
  \includegraphics{example-image-b}
  \caption{Un ejemplo de figura. Solo voy a poner una aunque podría poner muchas más.}
  \label{fig:unaFigura}
\end{figure}

\begin{figure}
  \centering
  \begin{figuredefinescaptionwidth}
    \includegraphics{example-image-c}
  \end{figuredefinescaptionwidth}
  \caption{Un ejemplo de figura con el título ajustado al ancho de la imagen.}
  \label{fig:otraFigura}
\end{figure}

\lstCppMakeShortInline|
En \texttt{macros.tex} se incluyen algunas ayudas. Por ejemplo, para mencionar
fácilmente programas y comandos como \Matlab o código en línea como
|'open("holamundo.txt")'|. En cualquier caso, gracias al paquete \texttt{listings}
se pueden crear \emph{floats} de listados de código. Y recuerda poner siempre
referencias a tus referencias bibliográficas. No hay nada como leer a grandes como
\cite{Torres2009}.
\lstDeleteShortInline|

\Blindtext
