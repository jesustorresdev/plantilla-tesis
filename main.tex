%%
%% main.tex - Memoria de tesis
%%
%%   Copyright 2009-2018 Jesús Torres <jmtorres@ull.es>
%%
%% Esta obra está bajo licencia Creative Commons Reconocimiento 4.0 Internacional
%%
\documentclass[b5paper,twoside,11pt]{scrbook}
\usepackage{local/tesis}
\usepackage{local/tesis-spanish}
%%
% Incluye:
%
%   Referencias: varioref, autoref[hyperef]
%   Algoritmos: algorithmic
%   Subfiguras: subcaption
%   Gráficos: graphicsx
%   Tablas: longtable, tabularx y booktabs
%       Define los tipos de columna C, L y R en modo matemático.
%%
% Mostrar marcas de corte
%\usepackage[a4,center,frame]{crop}
%%
% Mostrar el diagrama del layout de las páginas
%\usepackage{showframe}
%%
% Para poder definir algunas páginas apaisadas
%\usepackage{pdflscape}
%\usepackage{rotfloat}
%%
% Para generar un documento de ejemplo con el que probar la plantilla
\usepackage{blindtext}
\usepackage{mwe}

% Bibliografía
% Citas ordenadas en el orden en que aparecen en la bibliografía (sort).
% % En la primera cita aparecen los nombres de todos los autores (longnamesfirst)
% % \usepackage[longnamesfirst,sort]{natbib}
\usepackage[sort]{natbib}
\bibliographystyle{local/abbrvnat}
\bibpunct{(}{)}{;}{a}{,}{,}
\setbibpreamble{Las siguientes referencias bibliográficas se presentan en orden
alfabético por autor. Las referencias con más de un autor aparecen ordenadas
en base al primero de los mismos.\par\bigskip}

% Macros de uso común: operadores matemáticos, programas,
% listados de código, etc.
%%
%% macros.tex - Macros de uso común
%%
%%   Copyright 2009-2018 Jesús Torres <jmtorres@ull.es>
%%
%% Esta obra está bajo licencia Creative Commons Reconocimiento 4.0 Internacional
%%

% Operadores matemáticos adicionales
\DeclareMathOperator*{\argmax}{arg\,max}
\DeclareMathOperator*{\argmin}{arg\,min}
\DeclareMathOperator*{\diag}{diag}
\DeclareMathOperator*{\trace}{tr}
\DeclareMathOperator*{\var}{var}
\DeclareMathOperator*{\bigo}{\mathcal{O}}

% Otros elementos matemáticos
\newcommand{\relphantom}[1]{\mathrel{\phantom{#1}}}
\newcommand{\func}[1]{\mathnormal{#1}}
\newcommand{\mat}[1]{\boldsymbol{\mathrm{\MakeUppercase{#1}}}}
\newcommand{\spc}[1]{\mathbb{#1}}
\newcommand{\dist}[1]{\mathcal{#1}}
\renewcommand{\vec}[1]{\boldsymbol{#1}}
\newcommand{\UPSIGMA}{\boldsymbol{\Sigma}}

% Macros para ayudar a formatear los nombres de los programas
\usepackage{xspace}
\newcommand*{\program}[1]{{\ttfamily #1}}
\newcommand{\Blender}{\program{Blender}\xspace}
\newcommand{\Matlab}{\program{MATLAB}\xspace}
\newcommand{\Makehuman}{\program{MakeHuman}\xspace}

% Macros para formatear listados de código
\newcommand*{\lstCppMakeShortInline}[1]
  {\lstMakeShortInline[language=C++,basicstyle=\normalsize\ttfamily]#1}
\newcommand*{\lstMatlabMakeShortInline}[1]
  {\lstMakeShortInline[language=MATLAB,basicstyle=\normalsize\ttfamily]#1}
\newcommand*{\lstPythonMakeShortInline}[1]
  {\lstMakeShortInline[language=Python,basicstyle=\normalsize\ttfamily]#1}


%%
% Configuración de la salida en formato PDF
\pdfvariable imageresolution 300
\pdfvariable compresslevel 9
\hypersetup{
    ,bookmarksnumbered      =true
    ,pdfencoding            =auto,
    ,pdfborder              ={0 0 0},
    ,pdfdisplaydoctitle     =true,
    ,pdftitle               ={Título de la tesis},
    ,pdfauthor              ={Autor de la tesis},
    ,pdfsubject             ={Memoria de tesis doctoral},
    ,pdfkeywords            ={memoria, tesis, plantilla, latex}
}

%%
% Estructura del documento
\begin{document}

%% Parte inicial del documento
\frontmatter
\label{portada}\pdfbookmark{Tesis doctoral}{portada}
\titlehead{\includegraphics[height=1cm]{example-image-a} \\
  \vskip 1em
  Centro de la Universidad \\
  Departamento de la Universidad}
\subject{Tesis doctoral}
\title{Título de la tesis}
\author{Autor de la tesis}
\date{2018}
\maketitle

%%
%% autorizacion.tex - Autorización del tutor
%%
%%   Copyright 2009-2018 Jesús Torres <jmtorres@ull.es>
%%
%% Esta obra está bajo licencia Creative Commons Reconocimiento 4.0 Internacional
%%
\cleardoublepage
\thispagestyle{empty}
\hfill\begin{minipage}[t]{0.85\textwidth}\parindent 3em
D. [DIRECTOR DE LA TESIS], Doctor en [TITULACIÓN] y [CATEGORÍA] del
Departamento de [DEPARTAMENTO] de la [UNIVERSIDAD].
\null\vspace{\baselineskip}
CERTIFICA:

\vspace{\baselineskip}
que D. [DOCTORANDO], [TITULACIÓN], ha realizado bajo mi
dirección la presente Tesis, titulada ``[TÍTULO]'', para optar al
grado de Doctor por la [UNIVERSIDAD].

\vspace{\baselineskip}
Con esta fecha, autorizo la presentación de la misma.

\vspace{\baselineskip}
\hfill En LUGAR, a FECHA.

\vspace{\baselineskip}
\hfill\begin{tabular}{c}
El Director, \\\\\\\\
DIRECTOR DE LA TESIS
\end{tabular}
\end{minipage}

%%
%% dedicatoria.tex - Dedicatoria de la tesis
%%
%%   Copyright 2009-2018 Jesús Torres <jmtorres@ull.es>
%%
%% Esta obra está bajo licencia Creative Commons Reconocimiento 4.0 Internacional
%%
\cleardoublepage
\thispagestyle{empty}
\null\vskip 1cm
\textit{\raggedleft
  A XXX. \\*
  A YYY, VVV, WWW, \\*
  SSS y ZZZ.
  \vskip 4cm
  A XXXX y a YYYY, \\*
  VVVV, WWWW \\*
  y ZZZZ \\
}

%%
%% agradecimientos.tex - Memoria de la tesis
%%
%%   Copyright 2009-2010 Jesús Torres <jmtorres@ull.es>
%%
%% Esta obra está bajo licencia Creative Commons Reconocimiento 3.0 Unported
%%
\cleardoublepage
\thispagestyle{empty}
\addsec*{\protect\centering Agradecimientos}
\markboth{Agradecimientos}{}

Lorem ipsum dolor sit amet, consectetur adipiscing elit. Integer ut vestibulum
massa. Aliquam non magna sapien, a blandit mi. Fusce nec tellus tellus.
Vestibulum eu turpis eget ligula consequat ullamcorper. Cum sociis natoque
penatibus et magnis dis parturient montes, nascetur ridiculus mus. Duis risus
nisl, feugiat eget tempor a, lacinia eget nibh. Fusce sodales tristique
dapibus. Cras urna mi, iaculis in dapibus in, iaculis quis dui. Fusce
fringilla fermentum elementum. Phasellus fermentum quam at purus convallis et
placerat tortor rutrum. Vivamus sapien purus, placerat at sagittis at, lacinia
nec sapien. Donec consectetur aliquet mauris eu porta. Ut hendrerit faucibus
consequat. Nulla elit augue, sollicitudin hendrerit luctus quis, faucibus
vitae mauris. Phasellus ultrices feugiat purus vitae pellentesque. Aenean
vulputate sapien a sem rhoncus commodo. Maecenas eu elit felis. Maecenas
semper consequat risus sed hendrerit. Donec ac nisl ut eros laoreet accumsan
sed et arcu.

Morbi tempor tortor id urna facilisis gravida sagittis augue iaculis. Sed nec
lacus ac velit consequat semper et vitae odio. Cras pellentesque laoreet
venenatis. In ante augue, cursus sagittis semper quis, lacinia at eros. Morbi
dapibus, nisi non egestas aliquam, neque leo tempor orci, sit amet vehicula
tellus tortor in nisi. Integer posuere, quam quis posuere vestibulum, quam
purus tristique nisl, in venenatis justo tortor eu risus. Sed nec tincidunt
libero. Aenean molestie pretium dolor, vehicula consequat diam cursus vitae.
Aliquam erat volutpat. Nullam consectetur pharetra nisi id pulvinar. Maecenas
commodo dui id ante condimentum eu elementum magna euismod. Nulla leo erat,
pulvinar et mattis vitae, vestibulum quis mauris. Ut pretium pulvinar dolor,
vitae ornare tellus faucibus eu. Donec ullamcorper tincidunt nibh at euismod.
Donec dolor leo, mollis eget euismod vel, varius ac nunc. Vivamus iaculis,
lorem vitae hendrerit tristique, neque magna vehicula arcu, id pulvinar dolor
dolor a justo. Phasellus leo sapien, vehicula eu lacinia a, aliquet at leo.
Nunc tempus rhoncus ligula eget tincidunt. Quisque elementum tortor nec nisl
euismod adipiscing.

Mauris tortor odio, facilisis vitae varius quis, pharetra sit amet tortor.
Aliquam eu augue urna. Praesent lorem lectus, pretium suscipit fermentum
lobortis, fermentum vel massa. Pellentesque habitant morbi tristique senectus
et netus et malesuada fames ac turpis egestas. Quisque volutpat lacus sapien.
Pellentesque auctor orci et nisi auctor sagittis. Donec lobortis quam id arcu
ullamcorper nec ullamcorper nisi feugiat. Phasellus quis commodo nibh. Quisque
elit sem, imperdiet quis rutrum sed, sodales ac eros. Nulla facilisi. Praesent
nec diam at tellus varius ornare. Proin magna lacus, blandit a fermentum in,
volutpat ut elit. Vestibulum eget eros dui, a vulputate tellus. Cum sociis
natoque penatibus et magnis dis parturient montes, nascetur ridiculus mus.


\cleardoublepage
\pdfbookmark{\spanishcontentsname}{tableofcontents}
\tableofcontents

\cleardoublepage
\pdfbookmark{\listtablename}{listoftables}
\listoftables

\cleardoublepage
\pdfbookmark{\listfigurename}{listoffigures}
\listoffigures

\cleardoublepage
\pdfbookmark{\listalgorithmname}{listofalgorithms}
\listofalgorithms

%%
%% acronimos.tex - Memoria de la tesis
%%
%%   Copyright 2009-2010 Jesús Torres <jmtorres@ull.es>
%%
%% Esta obra está bajo licencia Creative Commons Reconocimiento 3.0 Unported
%%
\cleardoublepage
\chapter*{Índice de abreviaturas}\label{abreviaturas}
\pdfbookmark{Índice de abreviaturas}{abreviaturas}
\begin{acronym}[SLERP]
  \acro{CAE}{consectetur adipiscing elit}
  \acro{TELC}{turpis eget ligula consequat}
  \acro{UHFC}{ut hendrerit faucibus consequa}

\section*{Lorem ipsum dolor sit amet}
  \acro{LM}{Lorem}
  \acro{IP}{Ipsum}
  \acro{DL}{Dolor}
  \acro{AM}{Amet}
\end{acronym}

\acrodef{LMs}[LM]{Lorem}
\acrodef{IPs}[IP]{Ipsum}
\acrodef{DL}{Dolor}
\acrodef{AMs}[AM]{Amet}

%%
%% notacion.tex - Memoria de la tesis
%%
%%   Copyright 2009-2010 Jesús Torres <jmtorres@ull.es>
%%
%% Esta obra está bajo licencia Creative Commons Reconocimiento 3.0 Unported
%%
\chapter*{Símbolos y notación}\label{notación}
\pdfbookmark{Símbolos y notación}{notación}
\markboth{Símbolos y notación}{}
\begin{longtable}{rp{0.8\textwidth}}
  $a$               & escalar \\
  $|a|$             & valor absoluto de $a$ \\
  $\vec{a}$         & vector columna \\
  $a_i$             & $i$-ésimo elemento del vector $\vec{a}$ o del conjunto
    de escalares $\{a_n\}$  \\
  $\vec{1}$         & vector columna $[1,1,\ldots,1]^T$ \\
  $\|\vec{a}\|$     & norma euclídea de $\vec{a}$  \\
  $\vec{a}^T$       & traspuesta de $\vec{a}$ \\
  $\mat{A}$         & matriz  \\
  $\vec{a}_i$       & $i$-ésimo elemento de conjunto de vectores $\{\vec{a}_n\}$
    o $i$-ésima columna de la matriz $\mat{A}$ \\
  $a_{ij}$          & elemento en la fila $i$-ésima y columna $j$-ésima de
    la matriz $\mat{A}$ \\
  $\mat{I}$         & matriz identidad \\
  $\mat{A}^{-1}$    & inversa de la matriz $\mat{A}$ \\
  $\|\mat{A}\|_F$   & norma de Frobenius de la matriz $\mat{A}$ \\
  $\diag(\vec{a})$  & matriz diagonal cuyo $i$-ésimo elemento de la diagonal
    es $a_i$ \\
  $\det(\mat{A})$   & determinante de la matriz $\mat{A}$ \\
  $\trace(\mat{A})$ & traza de la matriz $\mat{A}$ \\
  $[\ldots]$        & vector o matriz \\
  $\{\ldots\}$      & conjunto o lista de elementos \\  
  $X, Y, Z$         & ejes de coordenadas en $\spc{R}^3$ \\
  $x, y$            & ejes de coordenadas en $\spc{R}^2$ \\
  $\spc{F}$         & espacio de características \\
  $\func{f}(x)$     & función $\func{f}$ en $x$ \\
  $\func{f}(x;p)$   & función $\func{f}$ en $x$ con parámetro $p$\\
  $\hat{a}$         & estimación de $a$ \\
  $\langle{a}\rangle$ & media del conjunto $\{a_n\}$ \\
  $\tilde{a}_i$     & $i$-ésimo elemento del conjunto $\{a_n\}$ al que se le ha
    restado la media de dicho conjunto \\
  $a^{(t)}$         & valor de $a$ en la iteración $t$ \\
  $\var(a)$         & varianza de $a$ \\
  $\dist{N}(\vec{\mu},\UPSIGMA)$  & distribución normal multivariable de media
    $\vec{\mu}$ y varianza $\UPSIGMA$ \\
  $\func{D}_{KL}(\dist{P}\|\dist{Q})$ & divergencia de Kullback-Liebler entre
    las distribuciones de probabilidad $\dist{P}$ y $\dist{Q}$
\end{longtable}


%%
%% introduccion.tex - Memoria de la tesis
%%
%%   Copyright 2009-2018 Jesús Torres <jmtorres@ull.es>
%%
%% Esta obra está bajo licencia Creative Commons Reconocimiento 4.0 Internacional
%%
\addchap{Introducción}

El objeto de este documento es mostrar algunos ejemplos de textos con formato.
Tenemos ~---guiones largos~--- en lugar de (paréntesis), <<comillas españolas>>
o latinas en lugar de ``comillas inglesas'' y \emph{cursiva} para los enfatizar
algo o para los términos en ingles. También tenemos acrónimos que si cuando se
usan la primera vez se muestra completos ~---por ejemplo, \ac{CAE}~--- pero la segunda vez y
posteriores no ~---por ejemplo, \ac{CAE}~---.

\Blindtext


%% Parte principal de la tesis
\mainmatter
%%
%% capitulo01.tex - Memoria de la tesis
%%
%%   Copyright 2009-2018 Jesús Torres <jmtorres@ull.es>
%%
%% Esta obra está bajo licencia Creative Commons Reconocimiento 4.0 Internacional
%%
\chapter{Capítulo 1}\label{ch:capitulo}

\section{Sección 1 del capítulo 1}\label{sec:seccion}

Dentro del documento se pueden describir algoritmos, que como son \emph{floats}
pueden aparecer en cualquier parte. Por eso se deben referenciar, como se puede ver
en \vautoref{alg:algoritmoRaro}, incluso a \autoref{alg:lin:output} y otras
líneas concretas. También podemos hablar por primera vez de \ac{LM} para que al volver
a usar \ac{LM} como acrónimo ocurra la magia. Y podemos incluir figuras ~---véase \vautoref{fig:unaFigura}~---.

\begin{algorithm}
\caption{Búsqueda de números pares.}\label{alg:algoritmoRaro}
\begin{algorithmic}[1]
\REQUIRE Un conjunto $v$ no vacío de enteros.
\ENSURE El primer elemento par de $v$.
\FOR{$i$ in $v$}\label{alg:lin:output}
\IF{$i$ es par}
\RETURN $i$
\ENDIF
\ENDFOR
\end{algorithmic}
\end{algorithm}

\begin{figure}
  \centering
  \includegraphics{example-image-b}
  \caption{Un ejemplo de figura. Solo voy a poner una aunque podría poner muchas más.}
  \label{fig:unaFigura}
\end{figure}

\begin{figure}
  \centering
  \begin{figuredefinescaptionwidth}
    \includegraphics{example-image-c}
  \end{figuredefinescaptionwidth}
  \caption{Un ejemplo de figura con el título ajustado al ancho de la imagen.}
  \label{fig:otraFigura}
\end{figure}

\lstCppMakeShortInline|
En \texttt{macros.tex} se incluyen algunas ayudas. Por ejemplo, para mencionar
fácilmente programas y comandos como \Matlab o código en línea como
|'open("holamundo.txt")'|. En cualquier caso, gracias al paquete \texttt{listings}
se pueden crear \emph{floats} de listados de código. Y recuerda poner siempre
referencias a tus referencias bibliográficas. No hay nada como leer a grandes como
\cite{Torres2009}.
\lstDeleteShortInline|

\Blindtext

\blinddocument

% Resetear los acrónimos. Se volverán a imprimir completos en el primero uso
%\acresetall

%%
%% conclusiones.tex - Memoria de la tesis
%%
%%   Copyright 2009-2018 Jesús Torres <jmtorres@ull.es>
%%
%% Esta obra está bajo licencia Creative Commons Reconocimiento 4.0 Internacional
%%
\addchap{Conclusiones}

\blindtext

\addsec{Líneas abiertas}

\blindtext


\appendix
%%%
%% apendiceA.tex - Memoria de la tesis
%%
%%   Copyright 2009-2010 Jesús Torres <jmtorres@ull.es>
%%
%% Esta obra está bajo licencia Creative Commons Reconocimiento 3.0 Unported
%%
\chapter{Apéndice A}
Lorem ipsum dolor sit amet, consectetur adipiscing elit. Vestibulum faucibus
sollicitudin facilisis. Donec sollicitudin augue eu lectus gravida mollis.
Nunc lacinia imperdiet posuere. Curabitur augue nulla, placerat sit amet
pellentesque a, venenatis id ante. Ut eget neque eros, sit amet ornare risus.
Quisque quis blandit tortor. Donec nunc justo, consectetur ac tempor eget,
dignissim ut arcu. Nam eu justo vel est imperdiet dignissim. Phasellus
vulputate augue nec est tincidunt fermentum. Sed hendrerit nisl sed turpis
blandit vestibulum. Vestibulum vitae lectus ac ligula varius iaculis. Maecenas
vitae lorem arcu, nec convallis leo. Nulla cursus, odio sit amet semper
hendrerit, odio magna dignissim massa, ut mattis justo felis at ante. Class
aptent taciti sociosqu ad litora torquent per conubia nostra, per inceptos
himenaeos.

\section{Sección 1 del apéndice A}\label{sec:apendice}
\lstCppMakeShortInline|
Nunc tincidunt fringilla dictum. In ac dolor et nibh viverra sodales a vitae
purus. Curabitur ultricies bibendum dolor eget lacinia. Duis fermentum neque
sit amet purus tempus in porta mauris posuere. Sed sit amet nulla odio. Nam
sit amet fermentum lorem. Mauris accumsan suscipit lorem. Suspendisse sed
nulla mi, id viverra tellus. Ut tristique egestas eros, ut molestie risus
aliquam nec. In aliquet, mi sit amet |'fringilla'| cursus, nisl velit
tincidunt lacus, in euismod lorem erat sit amet urna. Quisque ultrices luctus
odio sollicitudin |'malesuada'|.

\lstDeleteShortInline|

\blinddocument

%% Parte final del documento
\backmatter
\bibliography{references}

\end{document}
